\documentclass{article}
\usepackage[utf8]{inputenc}
\usepackage[italian]{babel}
\usepackage{hyperref}
\usepackage{enumitem}
\hypersetup{
    colorlinks=true,
    linkcolor=blue,
    urlcolor=blue,
    pdftitle={Guide per la mobilità all’estero presso l’UniMi},
    pdfauthor={Università degli Studi di Milano}
}

\title{Guide per la mobilità all’estero presso l’Università degli Studi di Milano}
\author{Servizio Studenti – UniMi}
\date{\today}

\begin{document}
\maketitle

\section{SEZIONE INTRODUTTIVA}
\subsection{Domanda e contesto}
L’utente ha richiesto: \textit{``riusciresti a trovare le pagine per andare all’estero''}.  
Di seguito trovi una panoramica delle pagine ufficiali dell’UniMi che trattano mobilità, borse di studio internazionali e maggiorazioni per periodi all’estero, insieme a una guida passo‑passo per accedere a tali opportunità.

\section{RISPOSTA DIRETTA}
\textbf{Le pagine principali da consultare sono:}
\begin{itemize}
    \item \href{https://www.unimi.it/it/studiare/frequentare-un-corso-post-laurea/dottorati-di-ricerca-phd/borse-e-tasse/maggiorazione-periodi-allestero}{Maggiorazione periodi all’estero (PhD)}  
    \item \href{https://www.unimi.it/it/studiare/frequentare-un-corso-post-laurea/dottorati-di-ricerca-phd/mobilita-internazionale-phd}{Mobilità internazionale PhD}  
    \item \href{https://www.unimi.it/it/studiare/borse-premi-mense-e-alloggi/borse-di-studio-internazionali}{Borse di studio internazionali}
\end{itemize}
Queste pagine contengono le informazioni necessarie per richiedere borse, maggiorazioni e partecipare a programmi di mobilità all’estero.

\section{GUIDA PASSO-PASSO}
\begin{enumerate}[label=\arabic*.]
    \item \textbf{Identifica il programma di tuo interesse}
    \begin{itemize}
        \item Se sei dottorando, consulta la pagina \href{https://www.unimi.it/it/studiare/frequentare-un-corso-post-laurea/dottorati-di-ricerca-phd/borse-e-tasse/maggiorazione-periodi-allestero}{Maggiorazione periodi all’estero}.
        \item Se vuoi partecipare a un progetto di mobilità internazionale (es. Marie Skłodowska‑Curie), vai su \href{https://www.unimi.it/it/studiare/frequentare-un-corso-post-laurea/dottorati-di-ricerca-phd/mobilita-internazionale-phd}{Mobilità internazionale PhD}.
        \item Se sei studente di primo anno o magistrale e cerchi borse di studio internazionali, visita \href{https://www.unimi.it/it/studiare/borse-premi-mense-e-alloggi/borse-di-studio-internazionali}{Borse di studio internazionali}.
    \end{itemize}
    \item \textbf{Verifica i requisiti di ammissione}
    \begin{itemize}
        \item Per la maggiorazione: essere beneficiario di una borsa di studio PhD, periodo all’estero non superiore a 12 mesi (18 mesi per i cicli 35°, 36°, 37°).
        \item Per la mobilità internazionale: essere iscritto al primo anno di dottorato e avere un progetto di ricerca concordato con un partner estero.
        \item Per le borse internazionali: rispettare i criteri di ammissione indicati nella borsa specifica (es. Excellence Scholarships).
    \end{itemize}
    \item \textbf{Richiedi l’autorizzazione al coordinatore}
    \begin{itemize}
        \item Compila e firma il modulo \textit{Richiesta maggiorazione estero PhD} (disponibile nella sezione \textit{Borse e tasse} della pagina di maggiorazione).
        \item Invia il modulo insieme all’autorizzazione del coordinatore alla Segreteria Dottorati (via InformaStudenti, sportello o consegna di persona).
    \end{itemize}
    \item \textbf{Invia la documentazione tramite InformaStudenti}
    \begin{itemize}
        \item Per la maggiorazione: entro il 3° giorno del mese successivo, carica gli attestati di frequenza (nome file: \texttt{COGNOME\_NOME\_MESE}) nella sezione \textit{Pagamenti borse di dottorato}.
        \item Per la mobilità internazionale: carica la documentazione richiesta (contratto di ricerca, lettera di accettazione del partner, ecc.) nella sezione dedicata di InformaStudenti.
    \end{itemize}
    \item \textbf{Attendi l’approvazione}
    \begin{itemize}
        \item La Segreteria Dottorati esaminerà la tua richiesta e ti comunicherà l’esito via e‑mail.
        \item Se approvata, la maggiorazione verrà accreditata nella mensilità successiva.
    \end{itemize}
    \item \textbf{Richiedi eventuali rimborsi spese}
    \begin{itemize}
        \item Se hai sostenuto spese di vitto, viaggio, alloggio o iscrizione a congressi, presenta la richiesta di rimborso alla segreteria amministrativa del dipartimento di afferenza, secondo il Regolamento delle Missioni e Rimborsi Spese.
    \end{itemize}
\end{enumerate}

\section{LINK E RISORSE}
\begin{itemize}
    \item \textbf{Maggiorazione periodi all’estero (PhD)}  
    \href{https://www.unimi.it/it/studiare/frequentare-un-corso-post-laurea/dottorati-di-ricerca-phd/borse-e-tasse/maggiorazione-periodi-allestero}{Maggiorazione periodi all’estero}
    \item \textbf{Mobilità internazionale PhD}  
    \href{https://www.unimi.it/it/studiare/frequentare-un-corso-post-laurea/dottorati-di-ricerca-phd/mobilita-internazionale-phd}{Mobilità internazionale PhD}
    \item \textbf{Borse di studio internazionali}  
    \href{https://www.unimi.it/it/studiare/borse-premi-mense-e-alloggi/borse-di-studio-internazionali}{Borse di studio internazionali}
    \item \textbf{InformaStudenti – Sezione Post‑Laurea}  
    \href{https://informasstudenti.unimi.it/}{InformaStudenti}
    \item \textbf{Ufficio Dottorati di ricerca e Scuola di Specializzazione}  
    \href{mailto:[email protected]}{Contatti via e‑mail}
\end{itemize}

\section{NOTE AGGIUNTIVE}
\begin{itemize}
    \item \textbf{Scadenze importanti:}
    \begin{itemize}
        \item Invio attestati di frequenza per la maggiorazione: entro il 28 luglio per il mese di luglio (per l’erogazione in agosto).
        \item Invio attestati di frequenza per il pagamento: entro il 3° giorno del mese successivo.
    \end{itemize}
    \item \textbf{Limiti di durata:}
    \begin{itemize}
        \item Maggiorazione: massimo 12 mesi (18 mesi per i cicli 35°, 36°, 37°).
    \end{itemize}
    \item \textbf{Tassazione:} Le borse di studio internazionali sono considerate reddito da lavoro dipendente e soggette a IRPEF (art. 50, comma 1, lettera c del TUIR).
    \item \textbf{Rimborsi spese:} Disponibili solo se i fondi sono disponibili e in conformità al Regolamento delle Missioni e Rimborsi Spese.
    \item \textbf{Euristica di ricerca:} Se hai bisogno di ulteriori dettagli su un programma specifico, consulta la sezione \textit{Borse e tasse} o contatta l’Ufficio Dottorati.
\end{itemize}

\end{document}
