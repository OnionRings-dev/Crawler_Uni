\documentclass{article}
\usepackage[utf8]{inputenc}
\usepackage[T1]{fontenc}
\usepackage[italian]{babel}
\usepackage{hyperref}
\usepackage{enumitem}

\title{Informazioni sul Programma Erasmus dell'Università degli Studi di Milano}
\author{Assistenza UniMi}
\date{\today}

\begin{document}
\maketitle

\section{Introduzione}
\textbf{Domanda dell'utente:} ``Parlami del programma Erasmus dell'università''.  
\textbf{Obiettivo della risposta:} Fornire una panoramica completa del programma Erasmus+ offerto dall'Università degli Studi di Milano (UniMi), con particolare attenzione ai requisiti, alle modalità di candidatura, alle scadenze e alle risorse utili per gli studenti interessati.

\section{Risposta Diretta}
Il programma \textit{Erasmus+} è la principale iniziativa europea per la mobilità studentesca e del personale.  
UniMi partecipa attivamente a questo programma, offrendo opportunità di studio all'estero a studenti di laurea triennale, magistrale e dottorato, nonché di mobilità per docenti e personale amministrativo.  
Le informazioni dettagliate, i moduli di candidatura e le scadenze sono disponibili sul portale UNIMIA e sul sito dell'Ufficio Internazionale dell'ateneo.

\section{Guida Passo-Passo}
\begin{enumerate}[label=\arabic*.]
    \item \textbf{Accedi al portale UNIMIA.}  
    \begin{itemize}
        \item Vai su \href{https://unimia.unimi.it}{https://unimia.unimi.it}.  
        \item Inserisci le credenziali di Ateneo (email \texttt{@unimi.it} o \texttt{@studenti.unimi.it}) e la password.  
        \item Se non possiedi ancora le credenziali, richiedile tramite il servizio \textit{Richiesta credenziali} presente nella stessa pagina.
    \end{itemize}
    \item \textbf{Naviga alla sezione \textit{Studenti Erasmus}.}  
    \begin{itemize}
        \item Una volta autenticato, cerca nella barra di navigazione la voce \textit{Studenti Erasmus} o \textit{Mobility}.  
        \item Se non trovi la voce, utilizza la funzione di ricerca interna del portale inserendo parole chiave come ``Erasmus'' o ``mobility''.
    \end{itemize}
    \item \textbf{Leggi le linee guida e i requisiti.}  
    \begin{itemize}
        \item Scarica il documento \textit{Guida Erasmus+ 2021-2027} (PDF) disponibile nella sezione.  
        \item Verifica di soddisfare i requisiti di ammissione: iscrizione attiva, buona media, eventuale conoscenza di lingua straniera.
    \end{itemize}
    \item \textbf{Compila il modulo di candidatura.}  
    \begin{itemize}
        \item Il modulo è disponibile in formato elettronico (PDF o form online).  
        \item Inserisci i dati personali, il corso di studi, la durata prevista e l'università ospitante (se già scelta).  
        \item Allegare eventuali documenti richiesti (certificato di iscrizione, curriculum, lettera di motivazione).
    \end{itemize}
    \item \textbf{Invia la candidatura.}  
    \begin{itemize}
        \item Carica il modulo compilato nella piattaforma di candidatura (solitamente \textit{Erasmus+ portal}).  
        \item Riceverai una mail di conferma con un numero di riferimento.
    \end{itemize}
    \item \textbf{Attendi l'approvazione.}  
    \begin{itemize}
        \item Il comitato di selezione dell'ateneo esamina la candidatura entro 4-6 settimane.  
        \item Se approvata, riceverai un \textit{Letter of Acceptance} e le istruzioni per la fase successiva.
    \end{itemize}
    \item \textbf{Organizza il viaggio e la burocrazia.}  
    \begin{itemize}
        \item Prenota volo, alloggio e assicurazione sanitaria.  
        \item Richiedi il \textit{Student Visa} (se necessario) e il \textit{Health Insurance} (es. \textit{European Health Insurance Card}).  
        \item Comunica all'Ufficio Internazionale eventuali modifiche di programma.
    \end{itemize}
    \item \textbf{Arrivo e registrazione all'università ospitante.}  
    \begin{itemize}
        \item Presenta la lettera di accettazione e il passaporto all'ufficio immatricolazioni.  
        \item Richiedi la registrazione dei crediti Erasmus+ per il riconoscimento a UniMi.
    \end{itemize}
\end{enumerate}

\section{Link e Risorse}
\begin{itemize}
    \item \href{https://unimia.unimi.it}{Portale UNIMIA – accesso studenti}  
    \item \href{https://www.unimi.it/it/ateneo/servizi/ufficio-internazionale}{Ufficio Internazionale – informazioni generali}  
    \item \href{https://www.unimi.it/it/ateneo/servizi/ufficio-internazionale/erasmus}{Sezione Erasmus – moduli e linee guida}  
    \item \href{https://erasmusplus.ec.europa.eu/}{Erasmus+ – sito ufficiale dell'UE}  
    \item \href{https://www.unimi.it/it/ateneo/servizi/ufficio-internazionale/contatti}{Contatti Ufficio Internazionale – email e telefono}  
\end{itemize}

\section{Note Aggiuntive}
\begin{itemize}
    \item \textbf{Scadenze:} Le date di apertura e chiusura delle candidature variano ogni anno.  Solitamente l'applicazione per il periodo 2025/2026 è aperta da settembre a dicembre 2024.  Verifica le scadenze specifiche sul sito dell'Ufficio Internazionale.  
    \item \textbf{Requisiti accademici:} È richiesta una media minima (di solito 24/30) e la partecipazione a un corso di lingua straniera.  
    \item \textbf{Finanziamento:} Il finanziamento Erasmus+ copre spese di viaggio, vitto e alloggio, ma non include spese di studio.  L'importo varia in base al paese ospitante.  
    \item \textbf{Crediti:} I crediti acquisiti all'estero vengono riconosciuti a UniMi tramite il sistema \textit{Erasmus+ credit transfer}.  Assicurati di richiedere la valutazione dei crediti prima della partenza.  
    \item \textbf{Assistenza:} Per dubbi o problemi durante la mobilità, contatta l'Ufficio Internazionale o il \textit{Student Support Service} dell'università ospitante.  
\end{itemize}

\end{document}
